\documentclass{beamer}
%\setbeameroption{show notes on second screen=right} % Both
\setbeameroption{hide notes} % Only slides
%\setbeameroption{show only notes} % Only notes
% Theme choice:
\usetheme{Boadilla}
\usepackage{tikz}
\usepackage{tikz-feynman}
\usepackage{animate}
\usepackage{amsmath}
\usepackage{cancel}
\usepackage{multicol}
\usepackage{physics}
\usepackage{graphicx}
\usepackage{amsfonts}
\usepackage{caption}
\captionsetup{singlelinecheck=false}
\usefonttheme[onlymath]{serif}
%\usepackage{hyperref}
%\hypersetup{colorlinks = true,linkcolor = black,filecolor= black,urlcolor= black}
%\usepackage{subcaption}
\usepackage[backend=bibtex,bibencoding=utf8,doi=false,isbn=false,url=false,eprint=false,indexing=false,style=authortitle]{biblatex}
\addbibresource{library}
\AtEveryBibitem{%
	\clearname{translator}%
	\clearlist{publisher}%
	\clearfield{pagetotal}%
}
\usepackage{MnSymbol,wasysym}
\usepackage[toc,page]{appendix}
\usetikzlibrary{snakes}
\usetikzlibrary{shapes.geometric, calc}
\author[Presenter: VCD Phuong]{\\
	Vo Chau Duc Phuong}
\institute[ICTP]{The Abdus Salam International Center for Theoretical Physics}
\title{Some Thing About Me}
\begin{document}
	\begin{frame}
\maketitle
	\end{frame}
\begin{frame}{Background}
\begin{tabular}{ll}
Name: & Phuong\\
Family name:& Vo Chau\\
University:& University of Science, VNU-HCM\\
Major:& Physics\\
Department:& Theoretical Physics\\
Graduation:& Honour class, top \(1^{st}\)\\
Location:& Ho Chi Minh city, Vietnam
\end{tabular}
\end{frame}
\begin{frame}
Internship: Institute of Physics, HCM city.\\
\textit{Supervisor: Dr. Huynh Thanh Duc.}
\begin{itemize}
\item Study and doing project on solid state physics - semiconductor.\\
\item Project: “Calculation of Shift Tensor in Transition Metal Dichalcogenide”.
\item Undergrad thesis: "Calculation of The Linear-Absorption Spectrum of An Ideal Two-dimensional
System of \(MoS_2\)".
\end{itemize}
Skills: Theory and numerical (FORTRAN, Python).
\end{frame}
\begin{frame}{Project:}
Name of project: \\\quad “Calculation of Shift Tensor in Transition Metal Dichalcogenide”.
Approach:
\begin{itemize}
	\item Second order perturbation in light-matter interaction\\
	\item Tight-binding model\\
	\item semiconductor Bloch equations
\end{itemize}
Methods:
\begin{itemize}
	\item Theoretical (in second quantization)\\
	\item Numerical (FORTRAN, PYTHON)
\end{itemize}
Result:\\
\begin{itemize}
\item From 3-band TB + SBEs \(\to\) shift current tensor for TMD monolayers.
\end{itemize}
Reported in: EIER 2024.
\end{frame}
\begin{frame}{Calculation of Shift Tensor in Transition Metal Dichalcogenide}
Why?
\begin{itemize}
\item Bulk photovoltaic (BPV) effect occurs in non-centrosymmetric materials, without the need of heterostructures or interfaces
\item BPV effect has the potential to overcome the Shockley–Queisser limit of photon–electricity conversion in a conventional p–n junction
\item TMD monolayers are semiconductors with a direct band gap, no center of inversion, Strong spin-orbit coupling leads to a spin splitting of hundreds meV.
\end{itemize}
\end{frame}
\begin{frame}
Based on the second perturbation of light-matter interaction:
$$\textbf{A}(t) = \sum_p A_{\omega_p} e^{-i\omega_p t} + c.c$$
Second-order current response:
\begin{equation*}
\textbf{J}^i_{shift} (\omega)= \sum_{j,k} \sigma^{i,j,k}_{shift}(\omega) A^{j*}_\omega A^k_\omega +c.c
\end{equation*}
Shift current tensor:
\begin{align}
	\sigma^{ijk}_{shift}(\omega) =& \frac{e^3}{L^2 \hbar^2 m^3} \sum_{c,v,\textbf{k}} \frac{p^k_{cv}(\textbf{k})}{(\varepsilon_c(\textbf{k}) - \varepsilon_v(\textbf{k}))/\hbar -\omega -i\gamma}\nonumber\\
	&\times \bigg[\sum_{\lambda \neq c} \frac{p^{j}_{v\lambda}(\textbf{k})p^{i}_{\lambda c}(\textbf{k})}{(\varepsilon_c(\textbf{k}) - \varepsilon_\lambda(\textbf{k}))/\hbar -\omega -i\gamma} \nonumber\\&\qquad - \sum_{\lambda \neq v} \frac{p^{i}_{v\lambda}(\textbf{k})p^{j}_{\lambda c}(\textbf{k})}{(\varepsilon_\lambda(\textbf{k}) - \varepsilon_v(\textbf{k}))/\hbar -\omega -i\gamma} \bigg] 
\end{align}
\end{frame}
\begin{frame}{Project also is my thesis:}
"Calculation of The Linear-Absorption Spectrum \\\hspace{4cm}of An Ideal Two-dimensional
System of \(MoS_2\)"
Approach:
\begin{itemize}
\item semiconductor Bloch Equations
\item Hatree-Fock Approximation to include the Coulomb interaction
\end{itemize}
Methods:
\begin{itemize}
	\item Theoretical (in second quantization)\\
	\item Numerical (FORTRAN run in parallel)
\end{itemize}
Result:\\
\begin{itemize}
\item Confirm the exciton binding energy in agree with experiment results.\\
\end{itemize}
Reported in front of:
\begin{center}
	Department of Theoretical physics (HCMUS-VNU)
\end{center}
\end{frame}
\begin{frame}{Current Status}
Diploma student in ICTP:
\begin{itemize}
\item First semester: All E (Excellent) grades.\\
\item Second semester (currently): studying DFT, Superconductivity, Advantage Numerical Method,...\\
\item Third semester (incoming): Doing thesis with prof. Natasha Stojic and available for other projects.
\end{itemize}
Research Interest:
\begin{itemize}
\item Quantum Hall Effect (Especially in Integer Quantum Hall effect, seft-taught).
\item Solid state physics, semiconductor physics (have experience).
\item Second phase-transition (Applying Landau-Ginzburg theory) in condensed matter (studying).
\end{itemize}
\end{frame}
\begin{frame}
\begin{center}
	Thank You For Your Interest.
\end{center}
\end{frame}
\end{document}