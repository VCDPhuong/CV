\documentclass[a4paper]{article}
\usepackage[a4paper, bindingoffset=0.2in, left=0.75in, right=0.75in, top=1in, bottom=1in, footskip=.25in]{geometry}
\usepackage{color}
\title{Statement of Purpose}
\author{Vo Chau Duc Phuong}
\begin{document}
	%\maketitle
	\parbox{0.5\textwidth}{Vo Chau Duc Phuong\\
		The Abdus Salam International Center for Theoretical Physics\\
		Trieste, Italy}
	\vspace{0.5cm}\\\null
	\quad Dear referee of the Doctoral school of Grenoble University, I am writing to submit my cover letter supporting my application for a funding for my PhD project named "Cavity-induced perturbations on the quantum Hall effect in graphene" at the Laboratoire de Physique et Modélisation des Milieux Condensés (LPMMC) in Grenoble, France, under the supervision of Drs Basko and Herviou. I want to devote myself to this PhD project since it directly connects my interest in topological materials and my previous experience.\\\null\quad 
	
	I have witnessed the transformative impact technology has on our lives, which has ignited my passion for its applications. Throughout high school, I participated in a physics program for gifted students that provided a strong foundation for my scientific career. During this time, I had numerous opportunities to attend various summer schools at other gifted high schools and universities, these experiences allowed me to meet enthusiastic teachers and students in physics, shaping my aspiration to become a physicist and engage in scientific research. I embarked on my academic journey at the University of Science, focusing on theoretical physics. During my internship at the Ho Chi Minh Center of Physics at IAMI under Dr. Huynh Thanh Duc, I investigated the properties of monolayer semiconductors, culminating in a presentation at EIER 2024, which sharpened my research skills and deepened my understanding of solid-state physics. My undergraduate thesis also supervised by Dr. Huynh involved simulating group-VI dichalcogenide monolayers, allowing me to compute the exciton binding energy in transition metal dichalcogenides (TMDs) and compare my findings with experimental data. In addition, I optimized Fortran and Python scripts to solve the Semiconductors Bloch Equation numerically. After completing my undergraduate as the top \(1^{st}\) student in the honours class, I was selected for the postgraduate program at the Abdus Salam International Center for Theoretical Physics (ICTP) in Trieste, Italy, which offered a comprehensive pre-PhD curriculum to further my education. This experience has been transformative, placing me in a diverse academic environment that has improved my research capabilities and boosted my confidence as a physicist. Studying at ICTP, I had the opportunity to participate in classes taught by well-known professors, and achieved overall excellent grades. Interacting with peers from diverse backgrounds has encouraged me to think critically and creatively about scientific challenges for the future of humanity.\\\null\quad
	
I am committed to teaching and mentorship, having engaged with the NES club in the undergraduate to learn from seniors and share my knowledge with freshmen. In my third year, I was one of the representatives for my school, and I achieved second place in the National Physics Olympiad for College Students. During my final year of undergraduate, I became my head department's teaching assistant in the Quantum Mechanics course for second-year students. Furthermore, I attended summer schools focused on solid-state physics and had the honour of designing and delivering lectures at the Math and Science Summer Program, which reinforced my expertise and fueled my passion for conveying complex concepts effectively. Moreover, I am actively involved in volunteer efforts at my university and the Ho Chi Minh City Youth Union. During the volunteer spring campaign, I helped organize a warm Lunar New Year celebration for underprivileged students and their families. Additionally, in the Green Summer campaign, I prepared and presented a scientific report to high school students to inspire them to pursue careers in science and to showcase how scientific advancements can enhance our everyday lives. The experiences mentioned above have shaped my philosophy regarding the role of scientists in society. Our responsibility extends beyond achieving technological advancements; ultimately, it is to enhance the quality of life for everyone in our community.\\\null\quad

I was first introduced to topology discussing to a senior math student, and was later impressed by the close relationship between the mathematical objects and several condensed matter models and properties, which in turn also led to important progress in the field of semiconductors. Having in mind the idea that semiconductors have changed the lives of humans, I raise an idea that another breakthrough might come with topological material in the future, which will decrease the power consumption of the semiconductor technology and enhance the potential for a better life. Therefore, I started learning the Quantum Hall Effect by myself, firstly for one of my courses in presentation skills, and secondly for my own interest in this future opportunity. Achieving a better understanding of the connection between edge modes and topology reinforced my interest in the field. Embarking on a career as a physicist has always been my aspiration, and I believe that completing a PhD is an essential step towards achieving that goal. However, the lack of funding opportunities in my country has presented a significant challenge. It was this obstacle that led me to explore alternative options, firstly moving to Trieste and then bringing me to the Doctoral school of Grenoble University. Your commitment to fostering innovation in quantum technology resonates deeply with me and has inspired my decision to pursue a PhD. Also, I was fortunate to connect with Drs Denis Basko and Loïc Herviou who have agreed to supervise my PhD project "Cavity-Induced Perturbations on the Quantum Hall Effect in Graphene", which directly connects my interest in topological materials and my previous experience with light-matter coupling in TMDs. This opportunity is not just a chance to advance my research interests; it's a vital pathway to ensuring my financial stability while making meaningful contributions to the field.\\\null\quad 

In the future, I plan to start my studies at the LPMMC lab in Grenoble in October 2025, which will ensure my study continue without any gap. During the first two years, I will establish a solid foundation for my research and build relationships that will support my work. I intend to find and attend conferences to gain insight into the current challenges scientists are facing in my field, which will help refine my thesis objectives. By the end of my first academic year, I aim to report my results at a top-tier international conference. I expect to wrap up my research by the middle of my third year, followed by one to two months for final revisions before finalization.\\\null\quad

	\quad Furthermore, I am keenly aware of France's esteemed reputation in higher education, especially regarding its influence on the scientific community in Vietnam. I am excited about the possibility of contributing to this legacy through my research, and I am eager to be part of a transformative educational experience in Grenoble that will equip me with the skills and knowledge to thrive in the world of quantum technology.\vspace{1cm}\\\null\quad
	Thank you for your reading and consideration,\\
	
	Vo Chau Duc Phuong
\end{document}